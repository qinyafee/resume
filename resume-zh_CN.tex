% !TEX TS-program = xelatex
% !TEX encoding = UTF-8 Unicode
% !Mode:: "TeX:UTF-8"

\documentclass{resume}
\usepackage{zh_CN-Adobefonts_external} % Simplified Chinese Support using external fonts (./fonts/zh_CN-Adobe/)
% \usepackage{NotoSansSC_external}
% \usepackage{NotoSerifCJKsc_external}
% \usepackage{zh_CN-Adobefonts_internal} % Simplified Chinese Support using system fonts
\usepackage{linespacing_fix} % disable extra space before next section
\usepackage{cite}

\begin{document}
\pagenumbering{gobble} % suppress displaying page number

\name{秦亚飞}

\basicInfo{
  \email{qinyafee@163.com} \textperiodcentered\
  \phone{(+86) 18501689646} \textperiodcentered\
  \faBirthdayCake{ 1990/07} \textperiodcentered}


\section{\faBook\ 自我评价}
% increase linespacing [parsep=0.5ex]
6年智驾开发/管理,量产经验丰富;熟悉slam、地图定位、标定;了解感知开发、自动标注业务

\section{\faCogs\ 专业技能}
% increase linespacing [parsep=0.5ex]
\begin{itemize}[parsep=0.5ex]
  \item 语言:C++ > Python
  \item 算法:熟悉滤波和优化的定位建图技术:鱼眼内参模型、图像/特征点矫正、多视图几何、预积分;图优化、EKF/UKF、粒子滤波;数据关联JPDA、Hungarian Match、车道线匹配
  \item 开源框架:ORB\_SLAM3/VINS-Mono/MSCKF/OpenVins
  \item 工具:ROS、PCL、OpenCV、g2o、Ceres Solver
  \item 优达学院无人驾驶工程师纳米学位(2017/10--2018/8)
\end{itemize}

\section{\faGraduationCap\  教育背景}
\datedsubsection{\textbf{浙江大学} \qquad 能源工程学系 \qquad  能源与环境系统工程 \qquad 本科}{2010/8--2014/6}

\section{\faUsers\ 工作经历}
\datedsubsection{\textbf{路特斯科技有限公司} \quad 算法软件部 \quad  定位建图算法负责人}{2022/7--今}
\begin{onehalfspacing}
  负责VE21高速高架NP/CP定位算法开发
  \begin{itemize}
    \item {搭建算法框架,带领团队开发全局/相对定位和3d车道线,以实现有/无hdmap下的NP/CP需求}
    \item 与上下游制定接口定义、KPI、状态预警
    \item 制定定位SIL方案,真值生产、数据切片/清洗、KPI评测、VIZ可视优化
  \end{itemize}
  负责lambda泊车感知建图的开发
  \begin{itemize}
    \item {基于行车框架,搭建泊车数据流;主导IPM拼接、停车位/freespace检测、建图等开发任务}
    \item 制定上下游TOPIC定义、数采录制、VIZ显示方案
    \item 支持4D标注,提供离线标定校验和点云建图算子
  \end{itemize}
  负责对lambda高速高架NP/CP定位算法的供应商管理
  \begin{itemize}
    \item {分析jira,与感知/规控/地图/传感器/系统等沟通,界定定位模块问题,并推动供应商制定优化方向,跟踪执行}
    \item 验收供应商定位KPI报告,把控关键指标
  \end{itemize}
\end{onehalfspacing}

\datedsubsection{\textbf{小鹏汽车科技有限公司} \quad 地图开发部 \quad  高精地图算法工程师}{2021/5--2022/7}
\begin{onehalfspacing}
  高速高架更新图层开发
  \begin{itemize}
    \item {负责车端更新图层模块开发,实现云端数据下载和地图错误修复功能,已量产交付;熟悉ADASISV3协议}
    \item 负责云端质检流程开发,支持地图数据几何/逻辑问题检查
  \end{itemize}

  城市更新图层开发
  \begin{itemize}
    \item 负责车道线生成算法,利用量产车众包上报的感知定位等数据,构建全局地图,修复地图错误或缺失区域
  \end{itemize}
\end{onehalfspacing}

\datedsubsection{\textbf{哈曼(中国)投资有限公司} \quad ADAS BU \quad 计算机视觉算法工程师}{2018/9--2021/5}
\begin{onehalfspacing}
  参与APA(辅助泊车)项目的开发,负责建图定位模块的开发
  \begin{itemize}
    \item 基于360环视系统,融合几何特征点、语义特征和轮速计信息,实现VWO视觉里程计
    \item 熟悉手眼标定原理,掌握里程计、环视系统标定方法
  \end{itemize}

  负责Camera/ LiDAR外参标定工具的开发
  \begin{itemize}
    \item 在线标定:实现基于边缘匹配的在线标定算法
    \item 离线标定:掌握基于AR tags的标定工具
  \end{itemize}

  参与TJA(交通拥堵辅助系统)的预研,负责融合跟踪模块的开发
  \begin{itemize}
    \item 基于Carla模拟生成激光雷达和毫米波雷达的目标检测列表,跟踪多目标轨迹
  \end{itemize}
\end{onehalfspacing}

\datedsubsection{\textbf{上海汽车集团商用车公司技术中心}  \quad  动力总成与电气化中心 \quad  系统工程师 }{2014/8--2018/9}
负责新能源关键系统的需求分析和技术开发;负责设计方案及变更发布、供应商管理

\begin{onehalfspacing}
  \begin{itemize}
    \item 开发电机及控制器、电子水泵/冷却风扇,配合整车ESP、防盗、远程监控、驻坡等功能的开发
    \item 推动解决各类NVH、EMC干扰、动力电池MSD击穿等复杂问题
  \end{itemize}
\end{onehalfspacing}

% Reference Test
%\datedsubsection{\textbf{Paper Title\cite{zaharia2012resilient}}}{May. 2015}
%An xxx optimized for xxx\cite{verma2015large}
%\begin{itemize}
%  \item main contribution
%\end{itemize}

\section{\faCode\ 其他}
\begin{itemize}[parsep=0.5ex]
  \item 多目VIO系统实现与动态初始化深度优化(VR头显设备)
\end{itemize}

%% Reference
%\newpage
%\bibliographystyle{IEEETran}
%\bibliography{mycite}
\end{document}
